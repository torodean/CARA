\chapter{Configuration}
\label{ch:configuration}
\pagestyle{fancy}

The CARA configuration system allows users to define how changelogs are generated and formatted. Configuration files provide a flexible and human-readable way to customize the behavior of the application without modifying the source code.

CARA reads configuration data from a plain text file, where each line defines a key-value pair separated by an equals sign (`=`). Lines beginning with `\#` are treated as comments and ignored. Empty lines and improperly formatted entries are also skipped during parsing.

The configuration system is implemented through the \texttt{Config} class, which is responsible for loading, parsing, and storing configuration values. Users can specify a configuration file at runtime via the \texttt{--config} flag. Once loaded, the configuration can be queried programmatically or used internally to guide CARA’s behavior. To use the default behavior for any configuration option, simply omit or comment out the option from the configuration file.

The expected format for configuration entries is:

\begin{lstlisting}[style=cppstyle]
key=value  # Optional comment.
\end{lstlisting}

This format provides clarity and simplicity, ensuring easy editing and version control. Each configuration key corresponds to a specific setting or feature within CARA, as documented in subsequent sections of this chapter.















\section{DATE\_FORMAT Configuration Option}

The \texttt{DATE\_FORMAT} option in the configuration file allows users to control the format in which commit dates are displayed in the generated changelog. This option directly maps to the \texttt{--date=format:<str>} flag in \texttt{git log}, enabling a wide range of formatting options based on user preferences or documentation standards. If this option is defined in the configuration file, CARA will invoke \texttt{git log} with the specified format string. Otherwise, Git's default date formatting is used. 

\begin{lstlisting}[style=cppstyle]
// Example Configuration
DATE_FORMAT=%Y-%m-%d
\end{lstlisting}

This formatted date is used in the changelog output entries. The header dates are more-so based on the \texttt{GROUP_BY} option (see section \ref{sec:groupby}).

\subsection*{Common Format Examples}
\begin{itemize}
	\item \texttt{\%Y-\%m-\%d} – 2025-07-01 (ISO standard)
	\item \texttt{\%d-\%m-\%Y} – 01-07-2025 (European)
	\item \texttt{\%B \%d, \%Y} – July 01, 2025 (Verbose)
	\item \texttt{\%a \%Y-\%m-\%d at \%H:\%M} – Tue 2025-07-01 at 14:30
\end{itemize}

For the full list of supported format tokens, refer to the \texttt{strftime(3)}\cite{strftime(3)} man page or Git's documentation on custom date formats.










\section{GROUP\_BY Configuration Option}
\label{sec:groupby}

The \texttt{GROUP\_BY} configuration value determines how commit entries in the generated changelog are grouped. Grouping provides structure to the changelog output by organizing commits into logical sections based on time. This option will set pre-formatted headers based on the groupings.

\subsection*{Accepted Values}
\begin{itemize}
	\item \texttt{day} --- Groups commits by individual date (e.g., \texttt{2025-07-01} (DayName)).
	\item \texttt{week} --- Groups commits by ISO calendar week (e.g., \texttt{2025-W27}).
	\item \texttt{month} --- Groups commits by calendar month (e.g., \texttt{2025-07} (MonthName)).
	\item \texttt{year} --- Groups commits by year (e.g., \texttt{2025}).
\end{itemize}

\subsection*{Default Behavior}
If this configuration value is not set, the default grouping is by \texttt{day}.

\subsection*{Usage}
This option affects the organization of the changelog. Each group is rendered as a section in the output, prefixed with a heading containing the grouping key (e.g., date or week label). Commits within each group are listed chronologically.









\section{OUTPUT\_ENTRIES Configuration Option}

The \texttt{OUTPUT\_ENTRIES} configuration option controls which fields are included for each commit entry in the generated changelog output. This value should be a space-separated list of field names, allowing flexible control over the formatting of each entry.

\subsection*{Valid Fields:}
\begin{itemize}
	\item \texttt{commit} -- Includes the full commit hash.
	\item \texttt{author} -- Includes the name of the author who made the commit.
	\item \texttt{message} -- Includes the commit message.
\end{itemize}

\subsection*{Special Value:}
\begin{itemize}
	\item \texttt{all} -- A shortcut that includes all of the above fields in the default format.
\end{itemize}

\subsection*{Example Values:}
\begin{itemize}
	\item \texttt{OUTPUT\_ENTRIES=commit message}
	\item \texttt{OUTPUT\_ENTRIES=author}
	\item \texttt{OUTPUT\_ENTRIES=all}
\end{itemize}

Each selected field will be printed in order, separated by colons and dashes according to default formatting. This allows for compact or verbose output depending on user preference.














\subsection{Filtering Configuration Options}

The changelog generator supports several filtering options to control which commit messages are included in the final output. These options allow users to exclude trivial or irrelevant commits and focus on meaningful changes.

\begin{description}
	\item[MIN\_WORDS] Specifies the minimum number of words required in a commit message for it to be included. If the message contains fewer words, it will be discarded. This option takes priority over \texttt{MIN\_CHARS} if both are set.
	
	\item[MIN\_CHARS] Specifies the minimum number of characters required in a commit message. If the message is shorter than this length, it will be excluded. Ignored if \texttt{MIN\_WORDS} is also set.
	
	\item[EXCLUDE\_KEYWORDS] A space-separated list of keywords. If any of these appear in a commit message (case-insensitive), the message will be excluded from the output.
	
	\item[INCLUDE\_KEYWORDS] A space-separated list of required keywords. Only commit messages containing at least one of these keywords (case-insensitive) will be included. If not set, this filter is ignored.
\end{description}

These options provide a flexible way to tailor the changelog content by filtering out uninformative or irrelevant commits.



